\chapter{Mehr Details!}
Mit den bisherigen Mitteln ist unsere Ausdrucksf"ahigkeit noch recht eingeschr"ankt, dieses Kapitel wird ihnen einige Werkzeuge in die Hand
geben um Ihre \lojb{bridi} mit mehr Leben zu f"ullen.

\section{\lojb{tanru}}
\lojb{tanru} \footnote{\lojb{tanru} werden teils auf Englisch auch etwas ungl"ucklich als ``metaphors'' bezeichnet} sind wir schon in einer unerw"unschten Variante begegnet, als mehrere \lojb{brivla} versehentlich zusammengelaufen sind. Richtig
verwendet sind \lojb{tanru} aber ein wichtiger Bestandteil der Artikulation auf Lojban; sie erm"oglichen es, Konzepte auszudr"ucken, f"ur die
bisher keine Worte existierten. Ein beliebtes Beispiel f"ur \lojb{tanru} ist ``Hundehaus''\footnote{Im Deutschen sind \lojb{tanru}-artige Konstrukte
recht h"aufig und erscheinen uns daher als nichts besonderes, andere Sprachen haben diese Tendenz nicht, daher sind \lojb{tanru} global betrachtet doch ein Fortschritt.}. Auf Lojban bildet man ein \lojb{tanru} indem man zwei \lojb{brivla} nebeneinander stellt. \lojb{gerku} ist das \lojb{gismu} f"ur ``Hund'', \lojb{zdani} das f"ur ``Haus''; ein Hundehaus w"are demnach ein \lojb{gerku zdani}. Genaugenommen bedeutet \lojb{gerku zdani}
 ``Hunde-artiges Haus'', es k"onnte genausogut ein hundef"ormiges Haus wie ein Hundehaus bezeichnen. \lojb{tanru} \emph{sind} mehrdeutig und 
 vom Kontext abh"angig, das ist gewollt. (Um trotzdem Begriffe etablieren zu k"onnen gibt es \lojb{lujvo}, dazu sp"ater mehr)
Die Platzstruktur dieses \lojb{tanru} entspricht der des zweiten \lojb{brivla} (auch \lojb{tertau} genannt). Den Pl"atzen des modifizierenden
\lojb{brivla} (dem \lojb{seltau}) k"onnen auch \lojb{sumti} zugewiesen werden, wie sehen wir sp"ater.

Wie w"urde man ein gro"ses Hundehaus nennen? \lojb{barda} bedeutet ``gro"s'', demnach liegt es nahe, etwas wie \lojb{barda gerku zdani} zu vermuten.
%hier ein beispiel bzgl grouping im deutschen einfuegen
Genau deshalb ist in Lojban die ``Klammerung'' (engl. grouping) festgelegt, die implizite Klammerung in Lojban ist linksgerichtet: \lojb{barda}
modifiziert \lojb{gerku} und \lojb{barda gerku} modifiziert \lojb{zdani}. Demnach ist ein \lojb{barda gerku zdani} ein Haus f"ur gro"se Hunde.
Lojban bietet mehrere M"oglichkeiten, diese Regel passend zu modifizieren. Eine M"oglichkeit ist, \lojb{bo} zu verwenden, es taucht an mehreren
Stellen in der Grammatik auf und dient dazu, Worte und Phrasen zusammenzubinden. In \lojb{barda gerku bo zdani} bindet das \lojb{bo} \lojb{gerku}
und \lojb{zdani} zusammen, so dass beide zusammen von \lojb{barda} modifiziert werden: Aus ((\lojb{barda gerku}) \lojb{zdani}) wird also
(\lojb{barda} (\lojb{gerku bo zdani})).

Eine andere, noch m"achtigere M"oglichkeit stellen \lojb{ke} und \lojb{ke'e} dar, sie entsprechen direkt den eben zur Erl"auterung verwendeten
Klammern. Gleichwertig zu der Variante mit \lojb{bo} w"are also \lojb{barda ke gerku zdani [ke'e]}, wobei das \lojb{ke'e} optional ist.

\lojb{tanru} k"onnen beliebig komplex sein, ein Beispiel daf"ur findet sich im letzten Abschnitt des Kapitels 5 der Referenzgrammatik.

\subsection{\lojb{sumti} einbinden}
Eine Maschine, die Hundeh"auser baut, k"onnte man demnach \lojb{gerku zdani zbasu bo minji} nennen, eleganter und \emph{lojbanischer} ist es jedoch,
 das $x_2$ von \lojb{zbasu}\footnote{$x_1$ setzt $x_2$ zusammen aus $x_3$} zu verwenden, da \lojb{zbasu} jedoch das \lojb{seltau} ist, k"onnen
wir dessen \lojb{sumti} nicht direkt angeben. Hier kommt das \lojb{cmavo} \lojb{be} ins Spiel, das es erm"oglicht, \lojb{sumti} in \lojb{selbri}
einzubinden (engl. linking \lojb{sumti}).
Die bessere L"osung ist also \lojb{zbasu be lo gerku zdani ku minji}. Standardm"a"sig beginnt \lojb{be} mit $x_2$, weitere \lojb{sumti} lassen
sich dann mit \lojb{bei} dazubinden. Eine Maschine, die Hundeh"auser aus Holzst"ucken baut w"are \lojb{zbasu be lo gerku zdani bei lo mudri spisa ku minji}, hier ist das \lojb{ku} nach \lojb{zdani} nicht n"otig, da das \lojb{bei} schon signalisiert, dass das \lojb{be}, das ja zu \lojb{zbasu} geh"ort, fortgesetzt wird.
Wenn Sie sich mehr Flexibilit"at w"unschen, so k"onnen Sie auch hier Platz-Tags verwenden: \lojb{zbasu be fi lo mudri spisa ku minji} ist eine
Maschine, die etwas aus Holzst"ucken zusammensetzt.

\subsection{\lojb{tanru}-Umkehrung}
Ein Mittel, das es manchmal erm"oglicht, das selbe wie mit \lojb{be} auf eine elegantere Art zu erreichen, ist \lojb{co}. Im ``Alltagslojban''
kommt es nicht sehr h"aufig vor, in lojbanischer Poesie jedoch h"aufiger.
Es vertauscht \lojb{seltau} und \lojb{tertau}, wenn es zwischen ihnen plaziert wird und alle \lojb{sumti} nach dem \lojb{selbri} werden in das
\lojb{seltau} eingebunden. Zur Erl"auterung sind folgende Aussagen vollkommen gleichwertig:
\begin{quote}
\lojb{ti zbasu be lo gerku zdani bei lo mudri spisa ku minji} \\
\lojb{ti minji co zbasu lo gerku zdani lo mudri spisa} \\
Dies ist eine Maschine, die Hundeh"auser aus Holzst"ucken zusammensetzt.
\end{quote}

\section{\lojb{BAI}-Tags}
Oftmals reichen die bisher beschriebenen Methoden nicht, all die Informationen unterzubringen, die man gerne unterbringen m"ochte. Ein weiterer,
m"achtiger weg sind \lojb{BAI}-Tags (auf Englisch auch ``modals'' genannt). Ein Beispiel f"ur einen \lojb{BAI}-Tag (Lojban kennt "uber 200!) ist
\lojb{gau} (abgeleitet von \lojb{gasnu}, $x_1$ verursacht $x_2$). Diese \lojb{cmavo} werden, "ahnlich wie Platztags vor \lojb{sumti} gestellt
und erzeugen einen weiteren Platz. \lojb{la tsezar morsi gau la brutus} bedeutet direkt "ubersetzt
\begin{quote}
Caesar ist tot, der Akteur ist Brutus
\end{quote}
Hieran kann man sowohl die ``Gefahr'' alsauch die N"utzlichkeit von \lojb{BAI}-Tags ableiten. Genauer w"are es n"amlich gewesen, \lojb{catra} zu
verwenden: $x_1$ t"otet $x_2$ durch $x_3$. In diesem Fall ist zwar noch relativ eindeutig, was mit \lojb{gau} gemeint ist, aber das muss nicht
immer der Fall sein. Andererseits k"onnen Ihnen \lojb{BAI}-Tags helfen, wenn sie z.B. nur \lojb{morji}, aber nicht \lojb{catra} kennen.

Der \lojb{BAI}-Tag \lojb{se pi'o} (dies ist ein sogenannter ``\lojb{cmavo} cluster'', also eine Ansammlung von \lojb{cmavo}) bedeutet ``mit Werkzeug..'' und ist von \lojb{pilno} abgeleitet ($x_1$ benutzt Werkzeug (im weitesten Sinne) $x_2$ f"ur $x_3$). Wie sie sehen haben \lojb{BAI}-Tags
eine starke "Ahnlichkeit mit den \lojb{gismu} von denen sie abgeleitet wurden: \lojb{pi'o} bedeutet ``Mit Benutzer..''.
Sollte Ihnen also einmal die Platzstruktur von \lojb{klama} entfallen und Sie wollen sagen, dass sie mit dem Auto fahren, k"onnten Sie also sagen: \lojb{mi klama sepi'o le karce}.
Nutzen Sie aber wenn m"oglich die festgelegten Pl"atze um ihren Zuh"orern das Herumr"atseln zu ersparen.

Nur was, wenn es keinen \lojb{BAI}-Tag gibt, der das ausdr"uckt, was Sie wollen? Oder wenn Sie sich gerade nicht daran erinnern k"onnen? (es gibt
vermutlich keinen Lojbanist, der \emph{alle} \lojb{BAI}-Tags auswendig kennt)
F"ur diesen Fall hat Lojban das \lojb{cmavo} \lojb{fi'o}, das aus einem beliebigen \lojb{selbri} einen vollwertigen \lojb{BAI}-Tag erzeugt, der das $x_1$ des \lojb{selbri} aufnimmt.
\begin{quote}
\lojb{mi klama le zarci fi'o sofybakni la alman} \\
Ich gehe zum Laden, involviert ist die sovietische Kuh ``Alma''.
\end{quote}
Wie sie sehen kann dieses \lojb{cmavo} sehr zur Verwirrung beitragen, doch es gibt auch sinnvolle Anwendungen:
\begin{quote}
\lojb{mi viska do fi'o kanla\footnote{$x_1$ ist ein Auge von $x_2$} le zunle} \\
Ich sehe dich. Mit Auge: Etwas linkes. \\
Ich sehe dich mit dem linken Auge.
\end{quote}

\section{Zeiten}
Das Zeitensystem von Lojban ist "au"serst umfangreich und vermag sehr viele Konzepte auszudr"ucken, die in anderen Sprachen nicht unbedingt
m"oglich sind (und erscheint deshalb manchmal, gegen die eigene Intuition zu arbeiten); das Kapitel zu Zeiten ist das l"angste in der
Referenzgrammatik. Deshalb kann hier nur ein Bruchteil davon erl"autert werden, sollten Sie mehr erfahren wollen werden konsultieren Sie Kapitel
10 der Referenzgrammatik.
