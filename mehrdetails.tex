\chapter{Mehr Details!}
Mit den bisherigen Mitteln ist unsere Ausdrucksf"ahigkeit noch recht eingeschr"ankt, dieses Kapitel wird ihnen einige Werkzeuge in die Hand
geben um Ihre \lojb{bridi} mit mehr Leben zu f"ullen.

\section{\lojb{tanru}}
\lojb{tanru} \footnote{\lojb{tanru} werden teils auf Englisch auch etwas ungl"ucklich als ``metaphors'' bezeichnet} sind wir schon in einer unerw"unschten Variante begegnet, als mehrere \lojb{brivla} versehentlich zusammengelaufen sind. Richtig
verwendet sind \lojb{tanru} aber ein wichtiger Bestandteil der Artikulation auf Lojban; sie erm"oglichen es, Konzepte auszudr"ucken, f"ur die
bisher keine Worte existierten. Ein beliebtes Beispiel f"ur \lojb{tanru} ist ``Hundehaus''\footnote{Im Deutschen sind \lojb{tanru}-artige Konstrukte
recht h"aufig und erscheinen uns daher als nichts besonderes, andere Sprachen haben diese Tendenz nicht, daher sind \lojb{tanru} global betrachtet doch ein Fortschritt.}. Auf Lojban bildet man ein \lojb{tanru} indem man zwei \lojb{brivla} nebeneinander stellt. \lojb{gerku} ist das \lojb{gismu} f"ur ``Hund'', \lojb{zdani} das f"ur ``Haus''; ein Hundehaus w"are demnach ein \lojb{gerku zdani}. Genaugenommen bedeutet \lojb{gerku zdani}
 ``Hunde-artiges Haus'', es k"onnte genauso gut ein hundef"ormiges Haus wie ein Hundehaus bezeichnen. \lojb{tanru} \emph{sind} mehrdeutig und 
 vom Kontext abh"angig, das ist gewollt. (Um trotzdem Begriffe etablieren zu k"onnen gibt es \lojb{lujvo}, dazu sp"ater mehr)
Die Platzstruktur dieses \lojb{tanru} entspricht der des zweiten \lojb{brivla} (auch \lojb{tertau} genannt). Den Pl"atzen des modifizierenden
\lojb{brivla} (dem \lojb{seltau}) k"onnen auch \lojb{sumti} zugewiesen werden, wie sehen wir sp"ater.

Wie w"urde man ein gro"ses Hundehaus nennen? \lojb{barda} bedeutet ``gro"s'', demnach liegt es nahe, etwas wie \lojb{barda gerku zdani} zu vermuten.
%hier ein Beispiel bzgl grouping im deutschen einfuegen
Genau deshalb ist in Lojban die ``Klammerung'' (engl. grouping) festgelegt, die implizite Klammerung in Lojban ist linksgerichtet: \lojb{barda}
modifiziert \lojb{gerku} und \lojb{barda gerku} modifiziert \lojb{zdani}. Demnach ist ein \lojb{barda gerku zdani} ein Haus f"ur gro"se Hunde.
Lojban bietet mehrere M"oglichkeiten, diese Regel passend zu modifizieren. Eine M"oglichkeit ist, \lojb{bo} zu verwenden, es taucht an mehreren
Stellen in der Grammatik auf und dient dazu, Worte und Phrasen zusammenzubinden. In \lojb{barda gerku bo zdani} bindet das \lojb{bo} \lojb{gerku}
und \lojb{zdani} zusammen, so dass beide zusammen von \lojb{barda} modifiziert werden: Aus ((\lojb{barda gerku}) \lojb{zdani}) wird also
(\lojb{barda} (\lojb{gerku bo zdani})).

Eine andere, noch m"achtigere M"oglichkeit stellen \lojb{ke} und \lojb{ke'e} dar, sie entsprechen direkt den eben zur Erl"auterung verwendeten
Klammern. Gleichwertig zu der Variante mit \lojb{bo} w"are also \lojb{barda ke gerku zdani [ke'e]}, wobei das \lojb{ke'e} optional ist.

\lojb{tanru} k"onnen beliebig komplex sein, ein Beispiel daf"ur findet sich im letzten Abschnitt des Kapitels 5 der Referenzgrammatik.

\subsection{\lojb{sumti} einbinden}
Eine Maschine, die Hundeh"auser baut, k"onnte man demnach \lojb{gerku zdani zbasu bo minji} nennen, eleganter und \emph{lojbanischer} ist es jedoch,
 das $x_2$ von \lojb{zbasu}\footnote{$x_1$ setzt $x_2$ zusammen aus $x_3$} zu verwenden, da \lojb{zbasu} jedoch das \lojb{seltau} ist, k"onnen
wir dessen \lojb{sumti} nicht direkt angeben. Hier kommt das \lojb{cmavo} \lojb{be} ins Spiel, das es erm"oglicht, \lojb{sumti} in \lojb{selbri}
einzubinden (engl. linking \lojb{sumti}).
Die bessere L"osung ist also \lojb{zbasu be lo gerku zdani ku minji}. Standardm"a"sig beginnt \lojb{be} mit $x_2$, weitere \lojb{sumti} lassen
sich dann mit \lojb{bei} dazubinden. Eine Maschine, die Hundeh"auser aus Holzst"ucken baut w"are \lojb{zbasu be lo gerku zdani bei lo mudri spisa ku minji}, hier ist das \lojb{ku} nach \lojb{zdani} nicht n"otig, da das \lojb{bei} schon signalisiert, dass das \lojb{be}, das ja zu \lojb{zbasu} geh"ort, fortgesetzt wird.
Wenn Sie sich mehr Flexibilit"at w"unschen, so k"onnen Sie auch hier Platz-Tags verwenden: \lojb{zbasu be fi lo mudri spisa ku minji} ist eine
Maschine, die etwas aus Holzst"ucken zusammensetzt.

\subsection{\lojb{tanru}-Umkehrung}
Ein Mittel, das es manchmal erm"oglicht, das selbe wie mit \lojb{be} auf eine elegantere Art zu erreichen, ist \lojb{co}. Im ``Alltagslojban''
kommt es nicht sehr h"aufig vor, in lojbanischer Poesie jedoch h"aufiger.
Es vertauscht \lojb{seltau} und \lojb{tertau}, wenn es zwischen ihnen platziert wird und alle \lojb{sumti} nach dem \lojb{selbri} werden in das
\lojb{seltau} eingebunden. Zur Erl"auterung sind folgende Aussagen vollkommen gleichwertig:
\begin{quote}
\lojb{ti zbasu be lo gerku zdani bei lo mudri spisa ku minji} \\
\lojb{ti minji co zbasu lo gerku zdani lo mudri spisa} \\
Dies ist eine Maschine, die Hundeh"auser aus Holzst"ucken zusammensetzt.
\end{quote}

\section{Abstraktionen}
Abstraktionen sind ein sehr m"achtiges Konzept in Lojban. Abstrakt gesprochen wandeln sie \lojb{bridi} in \lojb{selbri} um. \lojb{nu} ist wohl am h"aufigsten
verwendete Abstraktor (das \lojb{selma'o} der Abstraktionen tr"agt deshalb auch diesen Namen) und bedeutet:
\begin{quote}
$x_1$ ist das Ereignis von ...
\end{quote}

So bedeutet \lojb{nu mi prami do} ``$x_1$ ist das Ereignis, dass ich dich liebe''. Alleine stehend erscheint \lojb{nu} erst einmal wenig n"utzlich,
kombiniert man es allerdings mit einem \lojb{cmavo} wie \lojb{le}, er"offnen sich vielf"altige Anwendungsm"oglichkeiten (Dies ist so h"aufig, dass beide \lojb{cmavo} oftmals zusammen geschrieben werden, dies macht grammatikalisch aber keinen Unterschied). Einige \lojb{gismu} ``fordern'' sogar solche ``abstrakten \lojb{sumti}'' (engl. abstract \lojb{sumti}), so zum Beispiel \lojb{mukti} (Ereignis $x_1$ ist das Motiv
f"ur Ereignis $x_2$):
\begin{quote}
\lojb{le nu mi prami do mukti\footnote{M"oglicherweise wundern Sie sich, dass hier kein \lojb{cu} steht, aber es ist nicht zwingend n"otig, da das \lojb{bridi} der Abstraktion schon ein \lojb{selbri} hat, muss \lojb{mukti} zum ``Haupt-\lojb{bridi}'' geh"oren} le nu mi dunda le xrula do} \\
Das Ereignis ``Ich liebe dich'' ist die Motivation f"ur das Ereignis ``Ich gebe dir Blumen''. \\
Ich gebe dir Blumen, weil ich dich liebe.
\end{quote}
Abgesehen davon erzeugen Abstraktionen vollwertige \lojb{selbri} die auch in \lojb{tanru} verwendet werden k"onnen. Hier ist es oftmals erforderlich, sie explizit zu terminieren, was mit \lojb{kei} geschieht.

%passt das hier ueberhaupt?
\subsection{``Weil''}
Beim "Ubersetzen von und insbesondere nach Lojban bedarf das Wort ``weil'' besonderer Vorsicht. Betrachten wir die folgenden drei deutschen
S"atze:
\begin{itemize}
\item Es regnet, weil Wasser kondensiert.
\item Er kam ins Gef"angnis, weil er eine Straftat begangen hat.
\item Ich gebe dir Blumen, weil ich dich liebe.
\end{itemize}
Genauer betrachtet sind diese (kausalen) Zusammenh"ange verschiedener Natur: physikalisch, rechtlich, intrinsisch. Lojban dr"uckt diese drei
Klassen von Begr"undungen durch drei verschiedene \lojb{gismu} aus: \lojb{rinka}, \lojb{krinu} und \lojb{mukti}. Diese "Ubersetzung ist etwas ungl"ucklich, genauer lassen sich diese \lojb{gismu} so beschreiben:
\begin{description}
\item[\lojb{rinka}] Ein physikalischer oder ``mechanischer'' (keine ``willentliche'' Einflussnahme m"oglich) Zusammenhang.
\item[\lojb{mukti}] Eine durch etwas anderes motivierte Entscheidung.
\item[\lojb{krinu}] "Ahnlich wie \lojb{mukti}, folgt in der Regel aber Regeln.
\end{description}
Mit der Zeit werden Sie ein Gef"uhl daf"ur entwickeln, welches dieser \lojb{gismu} wann angebracht ist. Ein allgemeiner kausaler Zusammenhang
l"asst sich auch einfach durch \lojb{jalge} ausdr"ucken ($x_1$ ist das Ergebnis von $x_2$, beachten Sie, dass hier Ursache und Wirkung im
Vergleich mit den eben genannten \lojb{gismu} vertauscht sind!).

\subsection{Aussagen}
Ein weiterer wichtiger Abstraktor ist \lojb{du'u}, der soviel bedeutet wie ``$x_1$ ist die Aussage, dass ..., die durch $x_2$ ausgedr"uckt wird''.
So bedeutet zum Beispiel \lojb{mi djuno\footnote{$x_1$ kennt Fakt(en) $x_2$ zu Thema $x_3$ gem"a"s Erkenntnistheorie $x_4$} le du'u le cukta cu xunre} ``Ich wei"s, dass das Buch rot ist.''

Wof"ur dient nun das $x_2$ von einer \lojb{du'u}-Abstraktion? W"ahrend das $x_1$ den Fakt selbst darstellt, ist $x_2$ die (verbale) 
Repr"asentation dieser Aussage. Achten Sie in der \lojb{gismu}-Liste deshalb auf den Hinweis ``(\lojb{sedu'u})'' im Gegensatz zu ``(\lojb{du'u})''.
Ein Beispiel f"ur ein solches \lojb{gismu} ist \lojb{cusku} ($x_1$ sagt (nicht zwingend verbal) $x_2$ (\lojb{sedu'u}/text/\lojb{lu'e}) zu Zuh"orerschaft $x_3$ durch
Ausdrucksmedium $x_4$): \lojb{mi cusku le se du'u le cukta cu xunre} bedeutet ``Ich sage, dass das Buch rot ist''.

\subsection{Eigenschaften}
Der letzte wichtige Abstraktor, der zugleich Quelle f"ur mancherlei Verwirrung ist, ist \lojb{ka}: $x_1$ ist die Eigenschaft, die durch ...
ausgedr"uckt wird.
\begin{quote}
\lojb{la alis zmadu\footnote{$x_1$ "ubertrifft $x_2$ in Eigenschaft $x_3$ (ka/ni) um $x_4$} la bes le ka melbi\footnote{$x_1$ ist sch"on/angenehm f"ur $x_2$ in Eigenschaft $x_3$ (ka) nach Standard $x_4$} mi} \\
Alice "ubertrifft Beth in der Eigenschaft sch"on f"ur mich zu sein. \\
In meinen Augen ist Alice sch"oner als Beth.
\end{quote}

Beachten Sie jedoch, dass das, was wir alltagssprachlich als ``Eigenschaften'' beschreiben auf Lojban oftmals besser durch das \lojb{NU-cmavo za'i}, $x_1$
ist der Zustand, dass ... wahr ist. ``Gl"ucklichsein'' ist eher \lojb{le za'i gleki}.
Eigenschaften mit \lojb{ka} sind also eher sprachtechnischer Natur.

Ein weiteres \lojb{cmavo} das hier wichtig ist, ist \lojb{ce'u}, das den ``Fokus'' der Eigenschaft darstellt. (Wird \lojb{ce'u} nicht verwendet, so wird
%wie schaut es aus, wenn x1 belegt ist? Ich nehme mal an, dass es durchrutscht
angenommen, dass es den ersten nicht explizit belegten Platz des \lojb{selbri} der Abstraktion belegt.) Folgende Ausdr"ucke sind deshalb gleichwertig:
\begin{quote}
\lojb{le ka melbi mi} \\
\lojb{le ka ce'u melbi mi}
\lojb{le ka mi se melbi} \\
\lojb{le ka mi se melbi ce'u}
\end{quote}

%wie schon bei was? ich finde die stelle nicht mehr
Wie schon bei ... erscheint das \lojb{ce'u} hier "uberfl"ussig, f"ur komplexere Konstruktionen muss es aber manchmal verwendet werden.

\section{\lojb{BAI}-Tags}
Oftmals reichen die bisher beschriebenen Methoden nicht, all die Informationen unterzubringen, die man gerne unterbringen m"ochte. Ein weiterer,
m"achtiger weg sind \lojb{BAI}-Tags (auf Englisch auch ``modals'' genannt). Ein Beispiel f"ur einen \lojb{BAI}-Tag (Lojban kennt "uber 200!) ist
\lojb{gau} (abgeleitet von \lojb{gasnu}, $x_1$ verursacht $x_2$). Diese \lojb{cmavo} werden, "ahnlich wie Platztags vor \lojb{sumti} gestellt
und erzeugen einen weiteren Platz. \lojb{la tsezar morsi gau la brutus} bedeutet direkt "ubersetzt
\begin{quote}
Caesar ist tot, der Akteur ist Brutus
\end{quote}
Hieran kann man sowohl die ``Gefahr'' als auch die N"utzlichkeit von \lojb{BAI}-Tags ableiten. Genauer w"are es n"amlich gewesen, \lojb{catra} zu
verwenden: $x_1$ t"otet $x_2$ durch $x_3$. In diesem Fall ist zwar noch relativ eindeutig, was mit \lojb{gau} gemeint ist, aber das muss nicht
immer der Fall sein. Andererseits k"onnen Ihnen \lojb{BAI}-Tags helfen, wenn sie z.B. nur \lojb{morji}, aber nicht \lojb{catra} kennen.

Der \lojb{BAI}-Tag \lojb{se pi'o} (dies ist ein sogenannter ``\lojb{cmavo} cluster'', also eine Ansammlung von \lojb{cmavo}) bedeutet ``mit Werkzeug..'' und ist von \lojb{pilno} abgeleitet ($x_1$ benutzt Werkzeug (im weitesten Sinne) $x_2$ f"ur $x_3$). Wie sie sehen haben \lojb{BAI}-Tags
eine starke "Ahnlichkeit mit den \lojb{gismu} von denen sie abgeleitet wurden: \lojb{pi'o} bedeutet ``Mit Benutzer..''.
Sollte Ihnen also einmal die Platzstruktur von \lojb{klama} entfallen und Sie wollen sagen, dass sie mit dem Auto fahren, k"onnten Sie also sagen: \lojb{mi klama sepi'o le karce}.
Nutzen Sie aber wenn m"oglich die festgelegten Pl"atze um ihren Zuh"orern das Herumr"atseln zu ersparen.

Nur was, wenn es keinen \lojb{BAI}-Tag gibt, der das ausdr"uckt, was Sie wollen? Oder wenn Sie sich gerade nicht daran erinnern k"onnen? (es gibt
vermutlich keinen Lojbanist, der \emph{alle} \lojb{BAI}-Tags auswendig kennt)
F"ur diesen Fall hat Lojban das \lojb{cmavo} \lojb{fi'o}, das aus einem beliebigen \lojb{selbri} einen vollwertigen \lojb{BAI}-Tag erzeugt, der das $x_1$ des \lojb{selbri} aufnimmt.
\begin{quote}
\lojb{mi klama le zarci fi'o sofybakni la alman} \\
Ich gehe zum Laden, involviert ist die sovietische Kuh ``Alma''.
\end{quote}
Wie sie sehen kann dieses \lojb{cmavo} sehr zur Verwirrung beitragen, doch es gibt auch sinnvolle Anwendungen:
\begin{quote}
\lojb{mi viska do fi'o kanla\footnote{$x_1$ ist ein Auge von $x_2$} le zunle} \\
Ich sehe dich. Mit Auge: Etwas linkes. \\
Ich sehe dich mit dem linken Auge.
\end{quote}

\section{Zeiten}
Das Zeitensystem von Lojban ist "au"serst umfangreich und vermag sehr viele Konzepte auszudr"ucken, die in anderen Sprachen nicht unbedingt
m"oglich sind (und erscheint deshalb manchmal, gegen die eigene Intuition zu arbeiten); das Kapitel zu Zeiten ist das l"angste in der
Referenzgrammatik. Deshalb kann hier nur ein Bruchteil davon erl"autert werden, sollten Sie mehr erfahren wollen werden konsultieren Sie Kapitel
10 der Referenzgrammatik.

Lojban behandelt Raum und Zeit zumindest grammatikalisch gleich. Aus Anschaulichkeitsgr"unden werden wir zun"achst ``r"aumliche Zeiten'' behandeln.

\subsection{R"aumliche Zeiten}
Die Vorstellung bei r"aumlichen Zeiten (und, wie wir sp"ater sehen werden auch bei ``zeitlichen Zeiten'') ist, dass der Sprecher eine ``gedankliche Reise'' von sich zu dem Punkt des Geschehens beschreibt.
%stil?
Dies klingt abgehobener, als es eigentlich ist, deshalb nun ein Beispiel:
\begin{quote}
\lojb{le gerku zu'a zutse} \\
Der Hund [links] sitzt.\\
Links von mir sitzt ein Hund.
\end{quote}

\lojb{zu'a} geh"ort dem \lojb{selma'o FAhA} an und bedeutet ``links von ...''. Analog gibt es \lojb{ri'u} (rechts) und \lojb{ca'u} (vor) und viele mehr.
Diese \lojb{cmavo} lassen sich kombinieren um komplexere Reisen darzustellen.
\begin{quote}
\lojb{le gerku ri'u ca'u zutse} \\
Um den Ort zu erreichen, wo der Hund sitzt, bewege dich nach rechts und nach vorne. \\
Etwas rechts vor mir sitzt ein Hund.
\end{quote}
Da die gedanklichen Reisen vom Sprecher ausgehen, werden ``rechts'' und ``links'' "uber den Sprecher definiert; die Frage ``mein Rechts oder dein Rechts?'' werden Sie in Lojban nie stellen m"ussen.

Nun sagen diese \lojb{cmavo} nichts "uber die Entfernung aus, in der das Ereignis stattfindet, ob der Hund nun direkt neben mir oder tausende von
Kilometern entfernt sitzt, muss aus dem Kontext erschlossen werden, was im allgemeinen auch kein Problem darstellt. In Lojban ist es vollkommen
in Ordnung, vage zu sein. M"ochte man aber einem Touristen den Weg weisen ist es wom"oglich eine gute Idee, auch die Entfernung anzugeben.
Dies geschieht durch die \lojb{cmavo} \lojb{vi}, \lojb{va} und \lojb{vu} (erinnern Sie sich an \lojb{ti}, \lojb{ta} und \lojb{tu}!). Diese \lojb{cmavo} folgen auf die Richtung, die sie modifizieren.

\begin{quote}
\lojb{le gerku ri'u va ca'u zutse} \\
Ein St"uck rechts von mir und eine unspezifizierte Entfernung vor mir sitzt ein Hund.
\end{quote}

Es ist auch legitim, keine Richtung anzugeben:
\begin{quote}
\lojb{le gerku vu zutse} \\
Weit weg von mir sitzt ein Hund.
\end{quote}

\subsubsection{Zeiten als Tags}
%alle?
Alle Zeit-\lojb{cmavo} lassen sich auch als Tags verwenden:
\begin{quote}
\lojb{le gerku cu zutse zu'a le mlatu} \\
Der Hund sitzt links von der Katze.
\end{quote}
Auch Kombinationen von Zeit-\lojb{cmavo} lassen sich so verwenden.

Damit ist aber noch l"angst nicht alles zu r"aumlichen Zeiten gesagt, in der Referenzgrammatik findet sich die Beschreibung zu den weiteren
Aspekten wie z.B. Bewegung.

\subsection{Zeit}
Zeitlichen \lojb{cmavo} liegen exakt die gleichen Gedanken zugrunde wie r"aumlichen Zeiten (mit dem Unterschied, dass Zeit im Gegensatz zum
Raum eindimensional ist). \lojb{pu} und \lojb{ba} zeigen Vergangenheit bzw. Zukunft an, \lojb{ca} die Gegenwart. F"ur die zeitliche Entfernung existieren 
hier \lojb{zi}, \lojb{za} und \lojb{zu} (analog zu den \lojb{cmavo} \lojb{vi}, \lojb{va} und \lojb{vu}).
\begin{quote}
\lojb{mi ba klama} \\
Ich werde gehen. \\
\lojb{mi ba zi klama} \\
Ich werde in K"urze gehen.
\end{quote}
( Die deutsche "Ubersetzung dieses Beispiels hat die Implikation, dass der Sprecher \emph{jetzt} noch nicht geht, es
aber in Zukunft tun wird. Diese Implikation ist bei Lojban nicht gegeben, \lojb{mi ba klama} bedeutet nur, dass die Aussage ``Ich gehe'' zu einem
Zeitpunkt in der Zukunft wahr sein wird.)

Wie bei \lojb{VA} kann man auch hier die Richtung weglassen:
\begin{quote}
\lojb{mi za klama} \\
In mittlerer zeitlicher Entfernung ging ich/werde ich gehen.
\end{quote}
Diese Konstruktion mag f"ur uns un"ublich erscheinen ist aber nur eine logische fortf"uhrung der bisherigen "Uberlegungen.

\subsection{Intervalle und Aspekt}
Oftmals m"ochten wir die Dinge von denen wir sprechen als Prozesse, die ein bestimmtes Zeitintervall andauern, und nicht als punktf"ormige
Ereignisse darstellen. Um die L"ange eines solchen Intervalls anzugeben kennt Lojban die \lojb{cmavo} \lojb{ze'i}, \lojb{ze'a}, \lojb{ze'u},
die ein kurzes, mittellanges oder langes Zeitintervall anzeigen.
%soll hier ze'e erwaehnt werden?
%abbildung einfuegen/uebersetzen
Um solche Intervalle zu ``verankern'' gibt es die \lojb{cmavo} des etwas umfangreicheren \lojb{selma'o ZAhO}, eine "Ubersicht gibt Bild BLUB FIXME.

\subsection{Mehr Zeit}
Das Zeitsystem von Lojban kann noch einige Dinge mehr, die hier beschriebenen sind aber die am h"aufigsten verwendeten.
Zum Abschluss hier noch ein Beispiel, das einige der Zeit-\lojb{cmavo} verwendet, die in diesem Kapitel behandelt wurden:
\begin{quote}
\lojb{le ba mlatu cu puzi ze'u mo'u ri'uvu ca'uvi zutse} \\
%gottogott, stimmt das?
Die zuk"unftige Katze h"orte vor kurzem auf, eine lange Zeit weit rechts und ein wenig vor mir zu sitzen.
\end{quote}
