\chapter{Mehr Details!}
Mit den bisherigen Mitteln ist unsere Ausdrucksf"ahigkeit noch recht eingeschr"ankt, dieses Kapitel wird ihnen einige Werkzeuge in die Hand
geben um Ihre \lojb{bridi} mit mehr Leben zu f"ullen.

\section{\lojb{tanru}}
\lojb{tanru} \footnote{\lojb{tanru} werden teils auf Englisch auch etwas ungl"ucklich als ``metaphors'' bezeichnet} sind wir schon in einer unerw"unschten Variante begegnet, als mehrere \lojb{brivla} versehentlich zusammengelaufen sind. Richtig
verwendet sind \lojb{tanru} aber ein wichtiger Bestandteil der Artikulation auf Lojban; sie erm"oglichen es, Konzepte auszudr"ucken, f"ur die
bisher keine Worte existierten. Ein beliebtes Beispiel f"ur \lojb{tanru} ist ``Hundehaus''\footnote{Im Deutschen sind \lojb{tanru}-artige Konstrukte
recht h"aufig und erscheinen uns daher als nichts besonderes, andere Sprachen haben diese Tendenz nicht, daher sind \lojb{tanru} global betrachtet doch ein Fortschritt.}. Auf Lojban bildet man ein \lojb{tanru} indem man zwei \lojb{brivla} nebeneinander stellt. \lojb{gerku} ist das \lojb{gismu} f"ur ``Hund'', \lojb{zdani} das f"ur ``Haus''; ein Hundehaus w"are demnach ein \lojb{gerku zdani}. Genaugenommen bedeutet \lojb{gerku zdani}
 ``Hunde-artiges Haus'', es k"onnte genausogut ein hundef"ormiges Haus wie ein Hundehaus bezeichnen. \lojb{tanru} \emph{sind} mehrdeutig und 
 vom Kontext abh"angig, das ist gewollt. (Um trotzdem Begriffe etablieren zu k"onnen gibt es \lojb{lujvo}, dazu sp"ater mehr)
Die Platzstruktur dieses \lojb{tanru} entspricht der des zweiten \lojb{brivla} (auch \lojb{tertau} genannt). Den Pl"atzen des modifizierenden
\lojb{brivla} (dem \lojb{seltau}) k"onnen auch \lojb{sumti} zugewiesen werden, wie sehen wir sp"ater.

Wie w"urde man ein gro"ses Hundehaus nennen? \lojb{barda} bedeutet ``grosz'', demnach liegt es nahe, etwas wie \lojb{barda gerku zdani} zu vermuten.
%hier ein beispiel bzgl grouping im deutschen einfuegen
Genau deshalb ist in Lojban die ``Klammerung'' (engl. grouping) festgelegt, die implizite Klammerung in Lojban ist linksgerichtet: \lojb{barda}
modifiziert \lojb{gerku} und \lojb{barda gerku} modifiziert \lojb{zdani}. Demnach ist ein \lojb{barda gerku zdani} ein Haus f"ur gro"se Hunde.
Lojban bietet mehrere M"oglichkeiten, diese Regel passend zu modifizieren. Eine M"oglichkeit ist, \lojb{bo} zu verwenden, es taucht an mehreren
Stellen in der Grammatik auf und dient dazu, Worte und Phrasen zusammenzubinden. In \lojb{barda gerku bo zdani} bindet das \lojb{bo} \lojb{gerku}
und \lojb{zdani} zusammen, so dass beide zusammen von \lojb{barda} modifiziert werden: Aus ((\lojb{barda gerku}) \lojb{zdani}) wird also
(\lojb{barda} (\lojb{gerku (bo) zdani})).

Eine andere, noch m"achtigere M"oglichkeit stellen \lojb{ke} und \lojb{ke'e} dar, sie entsprechen direkt den eben zur Erl"auterung verwendeten
Klammern. Gleichwertig zu der Variante mit \lojb{bo} w"are also \lojb{barda ke gerku zdani [ke'e]}, wobei das \lojb{ke'e} optional ist.

\lojb{tanru} k"onnen beliebig komplex sein, ein Beispiel daf"ur findet sich im letzten Abschnitt des Kapitels 5 der Referenzgrammatik.
