\chapter{\lojb{bridi}}
Sich begr"u"sen zu k"onnen ist sicherlich wichtig, macht jedoch nicht den Kern eines Gespr"achs aus. Allgemeine S"atze im Deutschen finden sich in Lojban als \lojb{bridi} wieder.
Lassen Sie uns dazu einen deutschen Satz betrachten:
\begin{quote}
Ich fahre mit dem Auto zu Anna.
\end{quote}
In diesem Beispiel gibt es zum einen involvierte Objekte und Personen(``Ich'', ein Fahrrad und Anne) und etwas, dass diese Objekte miteinander verbindet, n"amlich, dass
eines f"ahrt, auf einem gefahren wird und zu einem hin gefahren wird.
Den entsprechenden Satz kann man wie folgt nach Lojban "ubersetzen: (Diese "Ubersetzung ist aus p"adagogischen Gr"unden so unsch"on, es geht auch einfacher, wie wir sp"ater sehen werden)
\begin{quote}
\lojb{mi klama la .anas. zo'e zo'e le karce}
\end{quote}
%sind es denn praepositionen?
\lojb{klama} entspricht hier dem ``fahren''. Im Deutschen werden die Pr"apositionen ``mit'' und ``zu'' verwendet um dem H"orer mitzuteilen, welches Objekt welche Aufgabe (Transportmittel, Ziel, ...) erf"ullt.
In Lojban ist dies f"ur \lojb{klama} festgelegt, die genaue Definition ist:
\begin{quote}
$x_1$ geht nach $x_2$ von $x_3$ "uber $x_4$ mit dem Transportmittel $x_5$
\end{quote}
(die Reihenfolge der ``Argumente'' $x_2$ und $x_3$ mag un"ublich erscheinen, ist aber sinnvoll, da die Argumente meist ihrer Wichtigkeit nach sortiert sind; wenn man 
sich fortbewegt, ist das Ziel in der Regel, nunja, das Ziel.)

\lojb{klama} dr"uckt hier die \textbf{Beziehung} der Argumente zueinander aus. So ein Wort fungiert als \lojb{selbri} (eine ad"aquate "Ubersetzung ist ``Pr"adikat''), die Argumente
(wie \lojb{mi}, \lojb{la .anas.}, usw.) nennt man \lojb{sumti}.
Bei den \lojb{sumti} handelt es sich um:
\begin{description}
\item[mi] Ein \lojb{cmavo} das ``Ich'' bedeutet
\item[la .anas.] \lojb{la} ist ein \lojb{cmavo}, der gesamte Ausdruck bedeutet soviel wie ``die, die Anna genannt wird''
\item[le karce] Auch \lojb{le} ist ein \lojb{cmavo}, der gesamte Ausdruck bedeutet ``etwas, das ich als Auto beschreibe'', oder einfach ``ein Auto''
(beachten Sie, dass \lojb{le karce} selbst keine Aussage "uber die Anzahl der Autos trifft, der Zuh"orer wird dies aber aus dem Kontext erschlie"sen)
\item[zo'e] \lojb{zo'e} dient dazu, die Pl"atze (engl. places) zu "uberspringen, die wir nicht angeben wollen. Die Bedeutung ist etwa ``hier steht etwas, aber es ist
gerade nicht wichtig, es zu erw"ahnen''. Es gibt weitaus elegantere Wege, Argumente zu "uberspringen, genaugenommen werden sie \lojb{zo'e} nur sehr selten begegnen, aber dazu
sp"ater mehr.
\end{description}

Wollte der Sprecher dem Zuh"orer nur mitteilen, dass er zu Anna f"ahrt, jedoch nicht womit, k"onnten wir also sagen:
\begin{quote}
\lojb{mi klama la .anas. zo'e zo'e zo'e}
\end{quote}
Das ist nat"urlich umst"andlich. Deshalb kann man \lojb{zo'e} am Ende einfach weglassen, wenn nur noch weitere \lojb{zo'e} folgen. Das vorherige Beispiel ist also identisch zu:
\begin{quote}
\lojb{mi klama la .anas.}
\end{quote}

W"orter, die als \lojb{selbri} fungieren k"onnen nennt man in Lojban \lojb{brivla}, zu den \lojb{brivla} geh"oren \lojb{gismu}, \lojb{lujvo} und \lojb{fu'ivla}. Sind Sie noch da?
Gut. Entwirren wir all diese Begriffe einmal.

\section{Grammatikalische Entwirrung}
Eine etwas subtile, aber ungemein wichtige Unterscheidung ist, was W"orter \textbf{sind} und was sie \textbf{tun}. Im Deutschen ist ``gehen'' ein Verb, das kann sich nicht "andern.
In einem Satz erf"ullt es hingegen die \textbf{Funktion} eines Pr"adikats. Durch eine leichte Ver"anderung wird aus ``gehen'' ``der Gehende'', dieses Wort kann in einem Satz z.B. die
Funktion des Subjekts erf"ullen (Der Vergleich hinkt ein wenig, hilft aber hoffentlich)i.

\lojb{selbri} und \lojb{sumti} sind \textbf{Funktionen}. \lojb{brivla} (und damit \lojb{gismu}, \lojb{lujvo} und \lojb{fu'ivla}) und \lojb{cmavo} sind Wort\textbf{arten}.

Nun ein kurzer "Uberblick "uber diese Wortarten.
\subsection{\lojb{gismu}}
\lojb{gismu} (auch als ``root words'', also Grundw"orter bezeichnet) sind der ``Kern'' der Sprache. Etwa 1300 sind definiert (die Definition von dem \lojb{gismu} \lojb{karce}
kennen Sie jetzt schon).
\lojb{gismu} sind leicht daran zu erkennen, dass sie immer f"unf Buchstaben haben und entweder dem Schema CVCCV (C ist ein Konsonant, V ein Vokal) oder CCVCV folgen. Beispiele f"ur beide Varianten w"aren sowohl
\lojb{gismu} selbst, als auch \lojb{karce}.
\lojb{gismu} allein decken schon einen nicht unwesentlichen Teil der Dinge ab, die man ausdr"ucken m"ochte. Mit einem Mechanismus der sp"ater erl"autert werden wird, lassen sich mit
ihnen unbegrenzt viele Konzepte darstellen.

\subsection{\lojb{lujvo}}
\lojb{lujvo} sind Worte, die urspr"unglich durch den eben genannten Mechanismus entstanden sind, deren Sinn aber in eine feste Form gegossen wurde. Um dies zu erl"autern greifen
wir ein wenig vor.

Ein Beispiel f"ur ein \lojb{lujvo} w"are \lojb{brivla}. Es setzt sich urspr"unglich aus \lojb{bridi} und \lojb{valsi} zusammen, die gemeinsam \lojb{bridi valsi} ergeben. (Dies nennt
man \lojb{tanru} und genau dies ist der vorhin bezeichnete Mechanismus)
Wird ein \lojb{tanru} h"aufig verwendet, wird es in der Regel zu einem \lojb{lujvo} auf einem klar definierten Weg abgek"urzt (die Umwandlung in ein \lojb{lujvo} hat noch weitere
Funktionen, die aber erst in dem entsprechenden Kapitel beschrieben werden sollen).

\lojb{bridi} bedeutet ``Satz'' oder ``Aussage'', ein \lojb{valsi} ist ein Wort. \lojb{brivla} bedeutet also ``Aussagewort''.

\subsection{\lojb{fu'ivla}}
\lojb{fu'ivla} sind Lehnw"orter (\lojb{fu'ivla} selbst ist ein \lojb{lujvo} das in etwa ``Kopiewort'' bedeutet). Manche Begriffe wie ``Spaghetti'' sind nur schwer zu "ubersetzen,
deshalb ist es manchmal sinnvoll, diese Begriffe zu "ubernehmen. Der genaue Funktionsweise wird ein eigenes Kapitel gewidmet werden (nur soviel sei schon verraten: Spaghetti
kann man z.B. als \lojb{cidjrspageti} bezeichnen).

\section{Pl"atze und was man damit machen kann}
\lojb{gismu} (mit wenigen Ausnahmen, wie \lojb{jutsi}) haben maximal f"unf definierte Pl"atze, meistens jedoch weniger (\lojb{remna}, $x_1$ ist ein Mensch, hat beispielsweise nur einen).
In einem \lojb{bridi} kann das \lojb{selbri} beliebig weit nach hinten geschoben werden, um das vorherige Beispiel aufzugreifen sind:
\begin{quote}
\lojb{mi klama la .anas. zo'e zo'e le karce \\
mi la .anas. klama zo'e zo'e le karce \\
mi la .anas. zo'e zo'e le karce cu klama}
\end{quote}
Vollkommen identisch. Das \lojb{cu} ist n"otig um zu verhindern, dass \lojb{karce} und \lojb{klama} ein \lojb{tanru} bilden. \lojb{cu} ist ein Zeichen, dass das \lojb{selbri} des
\lojb{bridi} folgt. Theoretisch kann man immer ein \lojb{cu} vor sein \lojb{selbri} legen, zwingend erforderlich ist das aber nur, wenn sich sonst
die Bedeutung des Satzes ver"andern w"urde. In diesem Fall h"atte man hier gar kein \lojb{selbri} und das Konstrukt am Ende w"urde ``ein autoartiger Gehender'' bedeuten.
Es steht ihnen v"ollig frei, wo sie ihr \lojb{selbri} plazieren, "ublich ist es jedoch, es zwischen $x_1$ und $x_2$ zu legen.

Steht das \lojb{selbri} am Anfang hat das \lojb{bridi} eine besondere Funktion, es ist ein ``observative'' (also ein Beobachtungssatz). Das $x_1$ verschwindet aus der Platzstruktur
(engl. ``place structure'') und wird als offensichtlich angenommen. So bedeutet
\begin{quote}
\lojb{fagri le zdani}
\end{quote}
``Feuer! Das Haus brennt!''. (W"ortlich: ``Ein Feuer mit dem Haus als Brennmaterial!'')

\subsection{Umwandlungen}
Angenommen wir wollten feststellen, dass ein Auto ein Fortbewegungsmittel ist. Es w"are m"oglich etwas zu sagen wie:
\begin{quote}
\lojb{zo'e klama zo'e zo'e zo'e le karce}
\end{quote}
Viel geschickter ist es jedoch, \lojb{klama} so zu ver"andern, dass das Fortbewegungsmittel in $x_1$ steht. Genau das leistet das \lojb{cmavo} \lojb{xe}.
Gleichbedeutend ist also die Aussage
\begin{quote}
\lojb{le karce cu xe klama}
\end{quote}
(das \lojb{cu} muss hier aus dem selben Grund wie oben verwendet werden)

Genaugenommen vertauscht \lojb{xe} $x_1$ und $_x5$, die Platzstruktur von \lojb{xe klama} ist also:
\begin{quote}
$x_1$ ist ein Fortbewegungsmittel um nach $x_2$ von $x_3$ "uber $x_4$ zu kommen, das von $x_5$ verwendet wird
\end{quote}

Analog existieren \lojb{se}, \lojb{te} und \lojb{ve}, die entsprechend $x_1$ und $x_2$, $x_1$ und $x_3$ oder eben $x_1$ und $x_4$ vertauschen. Diese \lojb{cmavo} geh"oren "ubrigens
zu dem \lojb{selma'o} \lojb{SE}.
Man kann diese Umwandlungen (engl. ``conversions'') auch miteinander kombinieren und etwas wie \lojb{te se xe klama} produzieren; hier werden zu erst $x_1$ und $x_5$ getauscht, dann
das neue $x_1$ und $x_2$, und dann $x_1$ und $x_3$, was letztenendes zu einer Platzstruktur der Form $x_3$,$x_5$,$x_2$,$x_4$,$x_1$ f"urht.
Da dies aber kaum einen anderen Zweck erf"ullt als den Zuh"orer zu verwirren gibt es noch eine weitere M"oglichkeit, die Reihenfolge von Pl"atzen umzuverteilen.

\subsection{Platztags}
Platztags\footnote{``tags'' l"asst sich beim besten Willen nicht angebracht "ubersetzen, am n"ahesten k"ame dem wohl ``Etikett''} (engl. ``place tags'') erm"oglichen es, Pl"atze
nach gutd"unken umzusortieren. Die f"unf Platztags sind \lojb{fa}, \lojb{fe}, \lojb{fi}, \lojb{fo} und \lojb{fu}. \lojb{fa} bedeutet ``jetzt kommt $x_1$'', die weiteren Platztags 
verhalten sich entsprechend.
Der "ubliche Weg um die Anfangs getroffene Aussage auszudr"ucken ist deshalb
\begin{quote}
\lojb{mi klama la .anas. fo le karce}
\end{quote}
So erspart man dem Zuh"orer das Mitz"ahlen von \lojb{zo'e} oder das Jonglieren mit Umwandlungen.
