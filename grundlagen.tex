\chapter{Grundlagen}
\section{Das Lojban-Alphabet}
Lojban kennt folgende Buchstaben:
\begin{quote}
' a b c d e f g i j k l m n o p r s t u v x y z
\end{quote}
Im Verglich zum deutschen Alphabet fehlen die Umlaute, h, w und q. Daf"ur ist der Apostroph ein vollwertiger Buchstabe. Diese
auf den ersten Blick etwas ungew"ohnliche Wahl begr"undet sich dadurch, dass der Apostroph nur in bestimmten Wortgruppen verwendet wird
und so im Schriftbild heraussteht.

\section{Die Aussprache}
Deutsche Muttersprachler haben es mit der Aussprache gl"ucklicherweise recht einfach. Zu aller erst ist anzumerken, dass in Lojban jeder Buchstabe auf genau eine Weise ausgesprochen wird (anders als im Deutschen, z.B. ist das s in "Suppe" stimmhaft, in "Maus" aber nicht).
Vom Deutschen unterscheiden sich nur die Buchstaben:
\begin{tabular}{|c|c|}
c && Wie ``Sch'' \\
j && Wie das g in ``Gara\bf{g}e'' \\ %hier pruefen ob das rendering passt
v && Wie das w in ``Wasser''\\
x && Wie das ch in ``Dach'' \\
y && Wie das e in ``M"ucke'' oder das a in ``about'' \\
z && Wie das s in ``Suppe'' (stimmhaft)
\end{tabular}
Der Apostroph wird als ``h'' (wie in ``Haus'') gesprochen.

\subsection{Diphtonge}
Diphtonge nennt man Kombinationen aus zwei Vokalen. Lojban kennt die Diphtonge
\begin{quote}
ai ei oi au ia ie ii io iu ua ue ui uu iy uy
\end{quote}
Diese werden gebunden ausgesprochen, so ist \lojb{ia} nicht ``Ie-ah'' sondern ``Ja''.

\subsection{Betonung}
Worte werden immer auf der vorletzten Silbe betont, auch das ist f"ur deutsche Muttersprachler in der Regel kein Problem. Abweichende
Ausspracheregelungen (beispielsweise wenn Namen nach Lojban "ubertragen werden) werden durch Gro"sbuchstaben angegeben,
beispielsweise bei dem Namen ``Pierre'': \lojb{pi,ER.}, hier liegt die Betonung auf dem ER. Das Komma dient der Silbentrennung und ist
ein Sonderzeichen das in Lojban (genau so wie die Grossschreibung) nur in Namen vorkommt.

\subsection{Pausen}
Im vorherigen Abschnitt hat sich noch ein Sonderzeichen eingeschlichen (das ist das Letzte, versprochen!): Der Punkt. Da in Lojban
Satzenden gesprochen werden (dazu kommen wir noch) hat der Punkt eine eigene Bedeutung, er stellt eine Pause dar. Strenggenommen
br"auchte man ihn nicht, da grammatikalisch klar ist, wann eine Pause stattfinden muss und wann nicht. Er dient daher nur zur Erinnerung.
Grammatikalisch muss eine Pause immer dann erfolgen, wenn
\begin{itemize}
\item ein Wort mit einem Vokal beginnt oder
\item ein Wort mit einem Konsonanten endet (wie \lojb{pi,ER.})
\end{itemize}
Diese F"alle sind erstaunlich selten, wie Sie sehen werden.

\section{Lojbanisieren von Namen und Sich-vorstellen}
Eines der ersten Dinge, die man in jedem Sprachkurs lernt, ist sich vorzustellen. Auch wir wollen mit dieser Tradition nicht
brechen. Mit den Informationen aus dem vorherigen Abschnitt sollte es ihnen m"oglich sein, Ihren Namen nach Lojban zu "ubertragen. (Sollten Sie ``Pierre'' hei"sen d"urfen Sie sich einen anderen Namen aussuchen)
Zu beachten ist jedoch, dass Namen auf Lojban (der Begriff hierf"ur ist \lojb{cmene}) immer auf einen Konsonanten enden m"ussen.
Dies kann entweder dadurch erreicht werden, dass man Vokale am Ende wegl"asst oder einen Konsonanten ("ublich sind s und n) anf"ugt.
Hier nun einige Beispiele:
\begin{tabular}{|c|c|c|}
Name & Lojbanisierung & Anmerkungen \\
\hline
Anna & \lojb{.anas.} & Hier musste ein Konsonant angef"ugt werden \\
Clara & \lojb{klar.} oder \lojb{klaras.} \\
Florian & \lojb{FLOrian.} & Um die richtige Betonung zu gew"ahrleisten muss die betonte Silbe hervorgehoben werden \\
Ibrahim & \lojb{.IBraxim.} oder \lojb{.IBra'im.} & In Namen kann der Apostroph vorkommen 
\end{tabular}

Das Funktionswort (im Folgenden \lojb{cmavo}) um sich vorzustellen ist \lojb{mi'e}, was so viel hei"st wie ``Ich hei"se''.
In diesem Zusammenhang ist es auch gut, \lojb{coi} zu kennen, was ``Hallo'' bedeutet.
Eine Konversation zwischen Ibrahim und Anna k"onnte so aussehen:
\begin{tabular}{|c|c|c|}
Lojban & Aussprache & "Ubersetzung \\
\hline
\begin{quote}
Ibrahim: \lojb{coi mi'e .IBraxim.}
Anna: \lojb{.i .ui coi .IBraxim. mi'e .anas.}
\end{quote}

&

\begin{quote}
I: Schoi Mi-he Ibrahim
A: I Ui \emph{(wie ``we'' auf Englisch)} Schoi Ibrahim Mi-he Anas
\end{quote}

&
\begin{quote}
I: Hallo! Ich hei"se Ibrahim
A: [Neuer Satz] [Freude] Hallo Ibrahim! Ich hei"se Anna
\emph{bzw.}
A: Hallo Ibrahim, sch"on dich zu sehen! Ich hei"se Anna
\end{quote}
\end{tabular}
Lesen Sie dieses (zugegebenerma"sen recht unspannende) Gespr"ach auch gerne laut (als Hilfestellung ist die Aussprache auch angegeben), wenn Sie die Beispiele in diesem Buch laut lesen werden Sie sich sehr schnell an die Aussprache gew"ohnen.

\section{Gef"uhle}
Im vorherigen Abschnitt wird Ihnen das Wort \lojb{.ui} aufgefallen sein. Wie die "Ubersetzung schon zeigte, handelt es sich um
ein Gef"uhls-/Einstellungswort (eine etwas ungl"uckliche "Ubersetzung des englischen Begriffs ``attitudinal'', deshalb soll dieser im
Folgenden verwendet werden).
Attitudinals tun genau das, was ihr Name Suggeriert: Sie sind eine M"oglichkeit den Gem"utszustand des Sprechers zu versprachlichen. Lojban hat eine F"ulle dieser Worte mit denen sich nahezu jede Kombination von Gef"uhlen ausdr"ucken l"asst.
Weiterhin werden Worte aus dieser Klasse auch zu anderen Zwecken verwendet, so zum Beispiel \lojb{za'a}, was den H"orer/Leser
wissen l"asst, dass die Aussage auf einer Beobachtung basiert (und z.B. nicht frei erfunden ist).
Steht ein Attitudinal am Anfang einer Aussage, so bezieht es sich auf die gesamte Aussage, ansonsten nur auf das Wort direkt davor.
