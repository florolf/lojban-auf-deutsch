\chapter{Grundlagen}
\section{Das Lojban-Alphabet}
Lojban kennt folgende Buchstaben:
\begin{quote}
' a b c d e f g i j k l m n o p r s t u v x y z
\end{quote}
Im Verglich zum deutschen Alphabet fehlen die Umlaute, h, w und q. Daf"ur ist der Apostroph ein vollwertiger Buchstabe. Diese
auf den ersten Blick etwas ungew"ohnliche Wahl begr"undet sich dadurch, dass der Apostroph nur in bestimmten Wortgruppen verwendet wird
und so im Schriftbild heraussteht.

\section{Die Aussprache}
Deutsche Muttersprachler haben es mit der Aussprache gl"ucklicherweise recht einfach. Zu aller erst ist anzumerken, dass in Lojban jeder Buchstabe auf genau eine Weise ausgesprochen wird (anders als im Deutschen, z.B. ist das s in "Suppe" stimmhaft, in "Maus" aber nicht).
Vom Deutschen unterscheiden sich nur die Buchstaben:
\begin{tabular}{|c|c|}
c && Wie ``Sch'' \\
j && Wie das g in ``Gara\bf{g}e'' \\ %hier pruefen ob das rendering passt
v && Wie das w in ``Wasser''\\
x && Wie das ch in ``Dach'' \\
y && Wie das e in ``M"ucke'' oder das a in ``about'' \\
z && Wie das s in ``Suppe'' (stimmhaft)
\end{tabular}
Der Apostroph wird als ``h'' (wie in ``Haus'') gesprochen.

\subsection{Diphtonge}
Diphtonge nennt man Kombinationen aus zwei Vokalen. Lojban kennt die Diphtonge
\begin{quote}
ai ei oi au ia ie ii io iu ua ue ui uu iy uy
\end{quote}
Diese werden gebunden ausgesprochen, so ist \lojb{ia} nicht ``Ie-ah'' sondern ``Ja''.
Von der offensichtlichen Aussprache unterscheiden sich nur:

\subsection{Betonung}
Worte werden immer auf der vorletzten Silbe betont, auch das ist f"ur deutsche Muttersprachler in der Regel kein Problem. Abweichende
Ausspracheregelungen (beispielsweise wenn Namen nach Lojban "ubertragen werden) werden durch Gro"sbuchstaben angegeben,
beispielsweise bei dem Namen ``Pierre'': \lojb{pi,ER.}, hier liegt die Betonung auf dem ER. Das Komma dient der Silbentrennung und ist
ein Sonderzeichen das in Lojban (genau so wie die Grossschreibung) nur in Namen vorkommt.

\section{Pausen}

