\chapter{Lojban?}
\section{Was ist Lojban?}
Lojban ist eine sogenannte ``konstruierte Sprache'', das hei\ss{}t, dass sie mit einem bestimmten Ziel entwickelt wurde und nicht auf
nat"urliche Weise entstanden ist. Der bekannteste Vertreter dieser Familie ist wohl Esperanto, welches mit dem Ziel, eine gemeinsame Sprache
f"ur alle Menschen zu schaffen entwickelt wurde. Dies geh"orte auch bei Lojban zu einem der Ziele, war aber nicht das Hauptziel. Am wichtigsten
war es den Entwicklern, eine Sprache zu schaffen, die
\begin{itemize}
\item Keine Grammatikalischen Mehrdeutigkeiten hat
\item Es erlaubt, mehr auszudr"ucken, als in nat"urlichen Sprachen m"oglich ist
\item Einen m"oglichst geringen ``kulturellen Stempel'' zu haben (Esperanto beispielsweise lehnt sich sehr an europ"aische Sprachen an)
\end{itemize}

Die ersten beiden Punkte bed"urfen noch einer Erkl"arung.
Ein Beispiel f"ur eine grammatikalische Mehrdeutigkeit im Deutschen ist der Satz ``Hans sieht Anna mit einem Fernrohr'', hier ist nicht klar, zu
wem der Satzteil ``mit dem Fernrohr'' geh"ort: Sieht Hans Anna \emph{durch} ein Fernrohr oder Anna, die eine Fernrohr tr"agt?
Derartige Probleme k"onnen in Lojban nicht auftreten, was aber nicht hei\ss{}t, dass keine Mehrdeutigkeiten auf der Bedeutungsebene vorliegen
k"onnen, die aus dem Zusammenhang erschlossen werden m"ussen. (Ohne diese w"are jegliche Kommunikation unertr"aglich aufw"andig)

Der Hintergrund des zweiten Punkts ist eher akademischer Natur, deshalb soll die Erl"auterung hier kurz gehalten werden. In der Linguistik gibt
es die Vermutung (die Sapir-Whorf-Hypothese), dass die Sprache, die wir sprechen (insbesondere unsere Muttersprache) die Art und Weise wie wir
denken formt (und unter Umst"anden auch beschr"ankt). Spricht man nun eine Sprache, die es einem erm"oglicht, mehr auszudr"ucken, so sollte man
auch in der Lage sein, mehr Konzepte gedanklich zu fassen.
% Haben wir hier ein besseres Beispiel?
Ein Beispiel in Lojban, das sich nur sehr gek"unstelt ins Deutsche "ubersetzen l"asst w"are \texttt{ko mo}, was so viel bedeutet wie ``Ich
befehle dir, etwas zu tun, wobei ich dich frage, was du tust''.

\section{Wieso sollte man Lojban lernen?}
Lojban hat bisher leider nicht viele Sprecher (dieses Buch m"ochte u.A. dazu beitragen, dass sich das "andert!), deshalb stellt sich die Frage,
wieso man es eigentlich lernen sollte.
