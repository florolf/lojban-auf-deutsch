\chapter{Lojban?}
\section{Was ist Lojban?}
Lojban ist eine sogenannte ``konstruierte Sprache'', das hei\ss{}t, dass sie mit einem bestimmten Ziel entwickelt wurde und nicht auf
nat"urliche Weise entstanden ist. Der bekannteste Vertreter dieser Familie ist wohl Esperanto, welches mit dem Ziel, eine gemeinsame Sprache
f"ur alle Menschen zu schaffen entwickelt wurde. Dies geh"orte auch bei Lojban zu einem der Ziele, war aber nicht das Hauptziel. Am wichtigsten
war es den Entwicklern, eine Sprache zu schaffen, die
\begin{itemize}
\item Keine Grammatikalischen Mehrdeutigkeiten hat
\item Es erlaubt, mehr auszudr"ucken, als in nat"urlichen Sprachen m"oglich ist
\item Einen m"oglichst geringen ``kulturellen Stempel'' zu haben (Esperanto beispielsweise lehnt sich sehr an europ"aische Sprachen an)
\end{itemize}

Die ersten beiden Punkte bed"urfen noch einer Erkl"arung.
Ein Beispiel f"ur eine grammatikalische Mehrdeutigkeit im Deutschen ist der Satz ``Hans sieht Anna mit einem Fernrohr'', hier ist nicht klar, zu
wem der Satzteil ``mit dem Fernrohr'' geh"ort: Sieht Hans Anna \emph{durch} ein Fernrohr oder Anna, die eine Fernrohr tr"agt?
Derartige Probleme k"onnen in Lojban nicht auftreten, was aber nicht hei\ss{}t, dass keine Mehrdeutigkeiten auf der Bedeutungsebene vorliegen
k"onnen, die aus dem Zusammenhang erschlossen werden m"ussen. (Ohne diese w"are jegliche Kommunikation unertr"aglich aufw"andig)

Der Hintergrund des zweiten Punkts ist eher akademischer Natur, deshalb soll die Erl"auterung hier kurz gehalten werden. In der Linguistik gibt
es die Vermutung (die Sapir-Whorf-Hypothese), dass die Sprache, die wir sprechen (insbesondere unsere Muttersprache) die Art und Weise wie wir
denken formt (und unter Umst"anden auch beschr"ankt). Spricht man nun eine Sprache, die es einem erm"oglicht, mehr auszudr"ucken, so sollte man
auch in der Lage sein, mehr Konzepte gedanklich zu fassen.
% Haben wir hier ein besseres Beispiel?
Ein Beispiel in Lojban, das sich nur sehr gek"unstelt ins Deutsche "ubersetzen l"asst w"are \texttt{ko mo}, was so viel bedeutet wie ``Ich
befehle dir, etwas zu tun, wobei ich dich frage, was du tust''.

\section{Wieso sollte man Lojban lernen?}
Beabsichtigt man eine Sprache zu lernen, mit der man m"oglichst viele Sprecher erreichen kann, sind Plansprachen sicherlich nicht die beste Wahl.
Dennoch gibt es einige Gr"unde, Lojban zu lernen. So unterscheidet sich das Sprachkonzept von Lojban sehr von den meisten anderen Sprachen;
eine Sprache mit anderen Ans"atzen als der eigenen Muttersprache zu lernen erm"oglicht einen neuen Blickwinkel auf diese. Weiterhin befinden sich die meisten Plansprachen (so auch Lojban), die letzlich viel j"unger sind als ihre nat"urlichen Gegenst"ucke, viel st"arker im Fluss. Lojbansprechern ist es so also m"oglich zur Gestaltung der Sprache beizutragen.

\section{Zur Terminologie}
Wie bereits gesagt unterscheidet sich Lojban in seinem Sprachkonzept von nat"urlichen Sprachen, deshalb gilt es in der Lojban-Community als \lojb{malglico}\footnote{``Verdammtes Englisch'', in unserem Fall also eher \lojb{maldotco}}
von ``S"atzen'', ``Verben'' und "Ahnlichem zu reden. Da Sie fr"uher oder sp"ater sowieso mit den Begriffen konfrontiert werden werden
werden in diesem Buch (soweit vorhanden) die Lojban-Begriffe f"ur grammatikalische Strukturen verwendet. Da die Hauptkonversationssprache der Lojban-Community (nach Lojban, nat"urlich) Englisch ist, sind auch einige englische Begriffe im Umlauf, die hier (so weit m"oglich) "ubersetzt werden. Im Anhang findet sich auch nochmals eine "Ubersicht der Entsprechungen der englischen Begriffe und der deutschen "Ubersetzungen.

\section{Von der Seele geschrieben}
Eine Sprache nur aus B"uchern zu lernen ist bei nat"urlichen Sprachen nahezu unm"oglich und bei Plansprachen schwer. W"ahrend
man bei nat"urlichen Sprachen in ein entsprechendes Land reisen kann, um diese zu "uben, ist dies bei Lojban nicht m"oglich.
%TODO: hier einen Link hinzufuegen
Am n"ahesten daran kommt aber der Lojban-Chat. Es ist auch f"ur Anf"anger sehr empfehlenswert diesen zu besuchen, auch, wenn es nur zum Mitlesen ist. (\lojb{coi}\footnote{``Hallo'' auf Lojban} sollte man der H"oflichkeit halber trotzdem sagen)
Es besteht dort die "Ubereinkunft, dass auf Lojban "uber alles geredet werden kann, in allen anderen Sprachen darf auch gesprochen werden,
aber dann sollten die Themen Lojban-bezogen sein. Deshalb ist dies auch der perfekte Ort, um Fragen zu stelle.n (z"ogern Sie bitte nicht, das zu tun, falls dieses Buch etwas offen l"asst!)
Die besten Chancen haben Sie nat"urlich auf Englisch, aber oft sind auch deutsche Lojbanisten anwesend.
Weiterhin ermutigt die Teilnahme an der Community zum weiterlernen, letztlich lernt man eine Sprache zum anwenden!
%TODO: Links!
Eine weitere Kommunikationsm"oglichkeit ist die Lojban Mailingliste.
