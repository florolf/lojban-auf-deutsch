\chapter{Neue Worte}
In diesem Kapitel wird genauer auf die schon erw"ahnten, aber nur sehr knapp erl"auterten Wortklassen \lojb{lujvo} und \lojb{fu'ivla} eigegangen
werden.

\subchapter{\lojb{lujvo}}
Wie bereits erl"autert sind \lojb{tanru} (absichtlich) mehrdeutig, was sowohl Vor- alsauch Nachteile hat. Der Hauptnachteil besteht nat"urlich
darin, dass ein Zuh"orer erraten muss, was mit einem bestimmten \lobj{tanru} gemeint ist. Um dem entgegenzuwirken existieren \lojb{lujvo}, die
morphologisch von \lojb{tanru} abgeleitet werden k"onnen und eine genau definierte Bedeutung haben. Ein Beispiel f"ur ein \lojb{lujvo} ist
%getreide? nachgucken!
\lojb{titnanba} ($x_1$ ist ein Kuchen aus Getreide $x_2$). Das zugeh"orige \lojb{tanru} ist \lojb{titla nanba}, das theoretisch
auch ein zuckerfarbenes Brot bezeichnen k"onnte.

\lojb{lujvo} setzen sich aus \lojb{rafsi} zusammen. Zu jedem \lojb{gismu} existieren mindestens zwei \lojb{rafsi}: Das \lojb{gismu} selbst und das \lojb{gismu} ohne
den abschlie"senden Vokal. Diese \lojb{rafsi} werden als 5- bzw. 4-Buchstaben-\lojb{rafsi} (engl. five/four letter \lojb{rafsi}) bezeichnet. 
H"aufig verwendeten \lojb{gismu} wurden bis zu drei 3-Buchstaben-\lojb{rafsi} zugewiesen. Diese \lojb{rafsi} haben eine der Formen CVV, CCV oder
CVC.
So hat \lojb{lojb} die \lojb{rafsi} \lojb{lob} und \lojb{jbo}. \lojb{bloti} ($x_1$ ist ein Boot um $x_2$ zu transportieren, angetrieben von $x_3$) hingegen hat \lojb{lot}, \lojb{blo} und \lojb{lo'i} (V'V z"ahlt in diesem Fall als VV).

%nochmal ueberpruefen
Demnach hat ein \lojb{lujvo} mehrere m"ogliche Formen, \lojb{jbopre} (die Bezeichnung f"ur Lojbanisten auf Lojban) ist von
\lojb{lojbo prenu} abgeleitet. M"ogliche \lojb{lujvo} w"aren deshalb z.B. noch \lojb{lojypre}, \lojb{lojbyprenu} und \lojb{jboprenu}. H"atten all diese Varianten verschiedene Bedeutungen w"are das insbesondere bei komplexeren \lojb{lujvo} sehr verwirrend, deshalb wurde festgelegt, dass alle Varianten eines \lojb{lujvo} die selbe Bedeutung haben.

Um ein \lojb{lujvo} zu bilden werden prinzipiell einfach die entsprechenden \lojb{rafsi} aneinander geh"angt. Hierbei m"ussen
jedoch die Lojban-Morphologieregeln (wie erlaubte/verbotene Konsonantenpaare) beachtet werden. Die genauen Regeln und eine
Schritt-f"ur-Schritt-Anleitung zum erzeugen von \lojb{lujvo} findet sich in Kapitel 4 der Regerenzgrammatik.

\subchapter{\lojb{fu'ivla}}
F"ur manche Begriffe ist es entweder schwer, ein passendes \lojb{lujvo} zu finden oder es w"urde beim Sprechen zu lange dauern,
sich eines zu "uberlegen. F"ur diesen Zweck gibt es \lojb{fu'ivla}, also Lehnw"orter (w"ortlich: Kopiew"orter).
\lojb{fu'ivla} werden in ``Stufen'' von 1 bis 4 eingeteilt, wobei die Einbettung in die Sprache mit der Stufe steigt.
F"ur das Wort ``Spaghetti'' w"aren diese \lojb{fu'ivla}:
\begin{description}
\item[Stufe 1] \lojb{[me] la'o zoi Spaghetti zoi}
\item[Stufe 2] \lojb{[me] la spagetis}
\item[Stufe 3] \lojb{cidjrspageti}
\item[Stufe 4] blub
\end{description}
