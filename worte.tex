\chapter{Neue Worte}
In diesem Kapitel wird genauer auf die schon erw"ahnten, aber nur sehr knapp erl"auterten Wortklassen \lojb{lujvo} und \lojb{fu'ivla} eigegangen
werden.

\section{\lojb{lujvo}}
Wie bereits erl"autert sind \lojb{tanru} (absichtlich) mehrdeutig, was sowohl Vor- alsauch Nachteile hat. Der Hauptnachteil besteht nat"urlich
darin, dass ein Zuh"orer erraten muss, was mit einem bestimmten \lojb{tanru} gemeint ist. Um dem entgegenzuwirken existieren \lojb{lujvo}, die
morphologisch von \lojb{tanru} abgeleitet werden k"onnen und eine genau definierte Bedeutung haben. Ein Beispiel f"ur ein \lojb{lujvo} ist
%getreide? nachgucken!
\lojb{titnanba} ($x_1$ ist ein Kuchen aus Getreide $x_2$). Das zugeh"orige \lojb{tanru} ist \lojb{titla nanba}, das theoretisch
auch ein zuckerfarbenes Brot bezeichnen k"onnte.

\lojb{lujvo} setzen sich aus \lojb{rafsi} zusammen. Zu jedem \lojb{gismu} existieren mindestens zwei \lojb{rafsi}: Das \lojb{gismu} selbst und das \lojb{gismu} ohne
den abschlie"senden Vokal. Diese \lojb{rafsi} werden als 5- bzw. 4-Buchstaben-\lojb{rafsi} (engl. five/four letter \lojb{rafsi}) bezeichnet. 
H"aufig verwendeten \lojb{gismu} wurden bis zu drei 3-Buchstaben-\lojb{rafsi} zugewiesen. Diese \lojb{rafsi} haben eine der Formen CVV, CCV oder
CVC.
So hat \lojb{lojb} die \lojb{rafsi} \lojb{lob} und \lojb{jbo}. \lojb{bloti} ($x_1$ ist ein Boot um $x_2$ zu transportieren, angetrieben von $x_3$) hingegen hat \lojb{lot}, \lojb{blo} und \lojb{lo'i} (V'V z"ahlt in diesem Fall als VV).

%nochmal ueberpruefen
Demnach hat ein \lojb{lujvo} mehrere m"ogliche Formen, \lojb{jbopre} (die Bezeichnung f"ur Lojbanisten auf Lojban) ist von
\lojb{lojbo prenu} abgeleitet. M"ogliche \lojb{lujvo} w"aren deshalb z.B. noch \lojb{lojypre}, \lojb{lojbyprenu} und \lojb{jboprenu}. H"atten all diese Varianten verschiedene Bedeutungen w"are das insbesondere bei komplexeren \lojb{lujvo} sehr verwirrend, deshalb wurde festgelegt, dass alle Varianten eines \lojb{lujvo} die selbe Bedeutung haben.

Um ein \lojb{lujvo} zu bilden werden prinzipiell einfach die entsprechenden \lojb{rafsi} aneinander geh"angt. Hierbei m"ussen
jedoch die Lojban-Morphologieregeln (wie erlaubte/verbotene Konsonantenpaare) beachtet werden. Die genauen Regeln und eine
Schritt-f"ur-Schritt-Anleitung zum erzeugen von \lojb{lujvo} findet sich in Kapitel 4, Abschnitt 11 der Regerenzgrammatik.

\section{\lojb{fu'ivla}}
F"ur manche Begriffe ist es entweder schwer, ein passendes \lojb{lujvo} zu finden oder es w"urde beim Sprechen zu lange dauern,
sich eines zu "uberlegen. F"ur diesen Zweck gibt es \lojb{fu'ivla}, also Lehnw"orter (w"ortlich: Kopiew"orter).
\lojb{fu'ivla} werden in ``Stufen'' von 1 bis 4 eingeteilt, wobei die Einbettung in die Sprache mit der Stufe steigt.
F"ur das Wort ``Spaghetti'' w"aren diese \lojb{fu'ivla}:
\begin{description}
\item[Stufe 1] \lojb{[me] la'o zoi Spaghetti zoi}
\item[Stufe 2] \lojb{[me] la spagetis}
\item[Stufe 3] \lojb{cidjrspageti}
\item[Stufe 4] \lojb{spageti}
\end{description}

\subsection{Stufe 1}
Ein \lojb{fu'ivla} der Stufe 1 entsteht durch die Verwendung des \lojb{cmavo} \lojb{la'o}, das bedeutet ``das, das nicht auf Lojban als .. bezeichnet wird''. Um den Namen eindeutig einzugrenzen muss auf das \lojb{la'o} ein Lojban-Wort folgen. Prinzipiell ist dieses Wort beliebig, es 
existiert aber die Konvention, die Sprache anzuzeigen, indem man den ersten Buchstaben des kulturellen \lojb{gismu} (wenn denn eines existiert)
verwendet. So w"urde man z.B. sagen: \lojb{la'o dy veilchen dy} oder: \lojb{la'o ly domus ly}. Im Zweifelsfall ist \lojb{zoi} "ublich.

\lojb{la'o} erzeugt ein \lojb{sumti}, m"ochte man ein solches \lojb{fu'ivla} als \lojb{selbri} verwenden, so kann man das \lojb{cmavo}
\lojb{me} verwendern: Es wandelt ein \lojb{sumti} in ein \lojb{selbri} um. Eine m"ogliche "Ubersetzung w"are ``$x_1$ ist spezifisch f"ur ...
in Aspekt $x_2$''.

\subsection{Stufe 2}
Um zur zweiten Stufe zu kommen wird der Name an die Morphologie von Lojban angepasst, wodurch die Trennworte "uberfl"ussig werden. Aus \lojb{la'o dy veilchen dy} w"urde so zu \lojb{la failxen} werden.
Hierbei bekommen wir aber immernoch ein \lojb{sumti}, will man ein \lojb{selbri}, so muss wieder \lojb{me} verwendet werden.

\subsection{Stufe 3}
% focus-speech?
Nun wird es interessanter. Ein \lojb{fu'ivla} der Stufe 3 besteht aus einem ''\lojb{rafsi}-Qualifizierer'' (engl. \lojb{rafsi} qualifier),
einem Bindekonsonant und einer angepassten Schreibweise des Namens wie bei Stufe 2. Bitte beachten Sie, dass das ``innere'' 
Wort mit einem Konsonanten beginnen und auf einen Vokal enden muss.

Der \lojb{rafsi}-Qualifizierer hat den Zweck, jemandem, der das zugrundeliegende Wort nicht kennt einen Anhaltspunkt zu geben,
um was es sich handeln k"onnte. Wenn man sichergehen will, dass das \lojb{fu'ivla} nicht zerf"allt, sollte man ein 4-Buchstaben-\lojb{rafsi} verwenden.

Der Bindekonsonant ist ``r'', sollte hierdurch ein doppeltes ``r'' entstehen (Doppelkonsonanten sind in Lojban verboten), so
muss stattdessen ein ``n'' verwendet werden. Sollte auch hier ein Doppelkonsonant entstehen (z.B. wenn das \lojb{rafsi} auf ``n''
endet und der Rest des \lojb{fu'ivla} mit ``r'' beginnt), verwendet man ein ``l''.

Beispiel: ``Veilchen'' wurde zu ``failxen''. Da dieses Wort schon mit einem Konsonanten beginnt muss es nur noch so angepasst
werden, dass es auf einen Vokal endet. Dazu kann man entweder das ``n'' weglassen oder einen weiteren Vokal anh"angen. In diesem
Beispiel soll die letztere Variante gew"ahlt werden: ``failxene''.
Als \lojb{rafsi}-Qualifizierer w"ahlen wir \lojb{xrul} von \lojb{xrula} ``Blume''.
Als Bindevokal muss demnach ein ``r'' verwendet werden. Das gesamte \lojb{fu'ivla} ist also: \lojb{xrulrfailxene}.

Solch ein \lojb{fu'ivla} ist ein vollwertiges \lojb{brivla}, dessen Platzstruktur wie die von \lojb{lujvo} festgelegt werden muss.

\subsection{Stufe 4}
Bei h"aufigen \lojb{fu'ivla} kann auch noch der Qualifizierer weggelassen werden, womit man ein \lojb{fu'ivla} der Stufe 4 erh"alt.
