\chapter{Neue Worte}
In diesem Kapitel wird genauer auf die schon erw"ahnten, aber nur sehr knapp erl"auterten Wortklassen \lojb{lujvo} und \lojb{fu'ivla} eigegangen
werden.

\subchapter{\lojb{lujvo}}
Wie bereits erl"autert sind \lojb{tanru} (absichtlich) mehrdeutig, was sowohl Vor- alsauch Nachteile hat. Der Hauptnachteil besteht nat"urlich
darin, dass ein Zuh"orer erraten muss, was mit einem bestimmten \lobj{tanru} gemeint ist. Um dem entgegenzuwirken existieren \lojb{lujvo}, die
morphologisch von \lojb{tanru} abgeleitet werden k"onnen und eine genau definierte Bedeutung haben. Ein Beispiel f"ur ein \lojb{lujvo} ist
%getreide? nachgucken!
\lojb{titnanba} ($x_1$ ist ein Kuchen aus Getreide $x_2$). Das zugeh"orige \lojb{tanru} ist \lojb{titla nanba}, das theoretisch auch ein mit 
Zucker bestreutes Brot bezeichnen k"onnte.

\lojb{lujvo} setzen sich aus \lojb{rafsi} zusammen. Zu jedem \lojb{gismu} existieren mindestens zwei \lojb{rafsi}: Das \lojb{gismu} selbst und das \lojb{gismu} ohne
den abschlie"senden Vokal. Diese \lojb{rafsi} werden als 5- bzw. 4-Buchstaben-\lojb{rafsi} (engl. four letter \lojb{rafsi}) bezeichnet. 
H"aufig verwendeten \lojb{gismu} wurden bis zu drei 3-Buchstaben-\lojb{rafsi} zugewiesen. Diese \lojb{rafsi} haben eine der Formen CVV, CCV oder
CVC.
